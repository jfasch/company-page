\subsection{\texttt{std::bind}}

% ---------------------------------------------------------------------------
\begin{frame}
  \frametitle{\texttt{std::bind}: Why?}

  \textbf{Why?} What's the problem? \linebreak
  \textbf{Answer:} 

  \begin{itemize}
  \item Hard to explain
  \item Best to see the problem first
  \item Let's start small, by simple example
  \end{itemize}

  \textbf{Problem: we have ...}

  \begin{itemize}
  \item Two dimensional points \texttt{(x,y)}
  \item A function to compute the distance between two points
  \end{itemize}

  \textbf{We want:}

  \begin{itemize}
  \item A function to compute the distance from \textit{origin}
    \texttt{(0,0)}
  \end{itemize}

\end{frame}

% ---------------------------------------------------------------------------
\begin{frame}[fragile]
  \frametitle{What We Have}

  \begin{block}{Point}
\begin{verbatim}
struct Point
{
  Point(double x, double y)
    : x(x), y(y) {}
  double x, y;
};
\end{verbatim}
  \end{block}

  \begin{block}{Distance}
\begin{verbatim}
double distance(Point p, Point q)
{
  return std::sqrt(
    std::pow(std::abs(p.x-q.x), 2) + 
    std::pow(std::abs(p.y-q.y), 2)
  );
}
\end{verbatim}
  \end{block}

\end{frame}

% ---------------------------------------------------------------------------
\begin{frame}[fragile]
  \frametitle{Retro C/C++}

  \begin{itemize}
  \item We have all that is needed
  \item Could easily define a small function
  \item $\to$ Problem solved
  \item \textit{But this would be soo retro!}
  \end{itemize}

  \begin{block}{Distance from Origin}
\begin{verbatim}
double distance_origin(Point p)
{
  return distance(p, {0,0});
}
\end{verbatim}
  \end{block}

\end{frame}

% ---------------------------------------------------------------------------
\begin{frame}
  \frametitle{The Real Problem}

  \textbf{Nothing is wrong with small functions}

  \begin{itemize}
  \item Compiler will inline them
    \begin{itemize}
    \item ... and optimize away entirely
    \end{itemize}
  \item Defined centrally (public header file?) for further reuse
  \end{itemize}

  \textbf{But...}

  \begin{itemize}
  \item What if they serve \textit{only one} purpose?
  \end{itemize}

  \begin{block}{Sample Problem}
    Compute the origin-distances of an array of points, and store
    those in an equally sized array of \texttt{double}!
  \end{block}
  
\end{frame}

% ---------------------------------------------------------------------------
\begin{frame}[fragile]
  \frametitle{Straightforward Implementation}

  Near the top of the implementation file ...

  \begin{block}{One-Time Function Definition}
\begin{verbatim}
static double distance_origin(Point p) {
    return distance(p, {0,0});
}
\end{verbatim}
  \end{block}

  And \textit{far down below}, in the implementation section ...
  
  \begin{block}{Location of use}
\begin{verbatim}
double distances_origin[sizeof(swarm)/sizeof(Point)];
std::transform(swarm, swarm+sizeof(swarm)/sizeof(Point), 
               distances_origin, 
               distance_origin);
\end{verbatim}
  \end{block}
  
\end{frame}

% ---------------------------------------------------------------------------
\begin{frame}[fragile]
  \frametitle{More Sample Problems}

  \begin{block}{Another Sample Problem}
    Compute the distances of an array of points from a given point,
    and store those in an equally sized array of \texttt{double}!
  \end{block}

  \textit{Possible solutions}: as many as there are different tastes
  around ...
  
  \begin{itemize}
  \item Lets write another stupid function, basically a copy of
    \texttt{distance\_origin} --- only with \texttt{(1,1)} instead of
    \texttt{(0,0)}
  \item Even better: lets generalize! \textit{Functors!}
    \textit{Function call operator!}
  \end{itemize}
  
\end{frame}

% ---------------------------------------------------------------------------
\begin{frame}[fragile]
  \frametitle{More Straightforward Implementations}

  \begin{block}{One-Time Functor Definition}
\begin{verbatim}
struct distance_point {
  distance_point(Point origin) : origin(origin) {}
  double operator()(Point p) const {
      return distance(p, origin);
  }
  Point origin;
};
\end{verbatim}
  \end{block}

  \begin{block}{Location of use}
\begin{verbatim}
double distances_origin[sizeof(swarm)/sizeof(Point)];
std::transform(swarm, swarm+sizeof(swarm)/sizeof(Point), 
               distances_point, 
               distance_point({1,1}));
\end{verbatim}
  \end{block}

\end{frame}

% ---------------------------------------------------------------------------
\begin{frame}
  \frametitle{Readability}

  \textit{Provided that the helper code is only used once ...}

  \begin{itemize}
  \item \textit{Readability} is inversely proportional to amount of
    code
  \item \textit{Number of bugs} is directly proportional to amount of
    code
  \item Helper implementation is nowhere near location of use
  \item \texttt{static} is the only keyword that enhances readability
  \end{itemize}

  \textit{Similar problem with many data structures and algorithms
    ...}

  \begin{itemize}
  \item Sorting criteria: \texttt{std::sort}, \texttt{std::map}, ...
  \item Predicates: \texttt{std::find\_if}, \texttt{std::equal}, ...
  \item Arbitrary adaptations where helper functions are needed
    \begin{itemize}
    \item Most prominent (although relatively useless nowadays):
      \texttt{std::for\_each}
    \end{itemize}
  \end{itemize}
  
\end{frame}

% ---------------------------------------------------------------------------
\begin{frame}[fragile]
  \frametitle{Introducing \texttt{std::bind} (1)}

  \textit{Best done by example ...}

  \begin{block}{}
\begin{verbatim}
void f(int a, int b) {
  std::cout << a << ',' << b << std::endl;
}
\end{verbatim}
  \end{block}

  \begin{columns}[t]

    \begin{column}{.6\textwidth}
      \begin{block}{Direct function call}
\begin{verbatim}
f(1, 2);
\end{verbatim}
      \end{block}
    \end{column}

    \begin{column}{.3\textwidth}
      \begin{block}{prints ...}
\begin{verbatim}
1,2
\end{verbatim}
      \end{block}
    \end{column}

  \end{columns}

\begin{verbatim}
\end{verbatim}

\textbf{What if} we need the functionality of \texttt{f(a, b)}, but
  are required to pass a \textit{callable} that taken no parameters?

\end{frame}

% ---------------------------------------------------------------------------
\begin{frame}[fragile]
  \frametitle{Introducing \texttt{std::bind} (2)}

  \textbf{In other words}, we need to create a function-like object
  that wraps \texttt{f(a,b)} that always calls \texttt{f} with, say,
  \texttt{a=1} and \texttt{b=2}.

  \begin{columns}[t]

    \begin{column}{.6\textwidth}
      \begin{block}{Hardcoded parameters}
\begin{verbatim}
auto bound = std::bind(f, 1, 2);
bound();
\end{verbatim}
      \end{block}
    \end{column}

    \begin{column}{.3\textwidth}
      \begin{block}{prints ...}
\begin{verbatim}
1,2
\end{verbatim}
      \end{block}
    \end{column}

  \end{columns}

  \begin{itemize}
  \item Alternative: manually write function adaptor (functor) that
    remembers parameters until called
  \item Origin: Boost (\href{www.boost.org}{www.boost.org})
  \end{itemize}

\end{frame}

% ---------------------------------------------------------------------------
\begin{frame}[fragile]
  \frametitle{Introducing \texttt{std::bind} (3)}

  \textbf{Routing parameters into arbitrary positions}:
  \texttt{std::placeholders}

  \begin{columns}[t]

    \begin{column}{.6\textwidth}
      \begin{block}{Hardcoding only second parameter}
\begin{verbatim}
auto bound = std::bind(f, 
       42, std::placeholders::_1);
bound(7);
\end{verbatim}
      \end{block}
    \end{column}

    \begin{column}{.3\textwidth}
      \begin{block}{prints ...}
\begin{verbatim}
42,7
\end{verbatim}
      \end{block}
    \end{column}

  \end{columns}

  \begin{columns}[t]

    \begin{column}{.6\textwidth}
      \begin{block}{Exchanging parameters}
\begin{verbatim}
auto bound = std::bind(f, 
       std::placeholders::_2, 
       std::placeholders::_1);
bound(1,2);
\end{verbatim}
      \end{block}
    \end{column}

    \begin{column}{.3\textwidth}
      \begin{block}{prints ...}
\begin{verbatim}
2,1
\end{verbatim}
      \end{block}
    \end{column}

  \end{columns}

\end{frame}

% ---------------------------------------------------------------------------
\begin{frame}[fragile]
  \frametitle{Applying \texttt{std::bind} (1)}

  \textbf{So how does this apply to our \texttt{std::transform}
    problem?}

  \begin{itemize}
  \item Readability: we want to eliminate those annoying extra helper
    functions
  \item Want to wrap existing \texttt{double distance(Point, Point)}
    which is similar in purpose but does not fit exactly
  \end{itemize}

  \begin{block}{What we have ...}
\begin{verbatim}
struct Point {...};
double distance(Point, Point);
\end{verbatim}
  \end{block}

  \begin{block}{What we want ...}
\begin{verbatim}
std::transform(swarm, swarm+sizeof(swarm)/sizeof(Point), 
               distances_point, 
               SOMETHING WHICH TAKES ONE POINT);
\end{verbatim}
  \end{block}
  
\end{frame}

% ---------------------------------------------------------------------------
\begin{frame}[fragile]
  \frametitle{Applying \texttt{std::bind} (2)}

  \begin{block}{Distances from origin}
\begin{verbatim}
std::transform(swarm, swarm+sizeof(swarm)/sizeof(Point), 
               distances_origin, 
               std::bind(distance, 
                  Point{0,0}, std::placeholders::_1));
\end{verbatim}
  \end{block}
  
  \begin{block}{Distances from any point}
\begin{verbatim}
// this is exactly the same as above
\end{verbatim}
  \end{block}

  \begin{columns}[t]
  \end{columns}

  \textbf{Summary}

  \begin{itemize}
  \item Readability: what remains unreadable is only the language
    itself
  \item Have to get used to \texttt{std::bind}
  \end{itemize}

\end{frame}

% ---------------------------------------------------------------------------
\begin{frame}[fragile]
  \frametitle{\texttt{std::bind} vs. Lambda}

  \textbf{Lambdas are usually a better alternative} ...

  \begin{block}{}
\begin{verbatim}
std::transform(swarm, swarm+sizeof(swarm)/sizeof(Point), 
               distances_origin, 
               [](Point p) { return distance({0,0}, p); });
\end{verbatim}
  \end{block}

  \begin{block}{A more advanced exercise}
    Use \texttt{std::sort} to sort an array of points by their
    distance to a given point.
  \end{block}
  
\end{frame}

% ---------------------------------------------------------------------------
\begin{frame}[fragile]
  \frametitle{A Bigger Picture: Types}

  \textbf{What about types?}

  \begin{itemize}
  \item Goal is to have \textit{no runtime overhead}
  \item $\implies$ \textit{Late binding (polymorphism)} ruled out
  \item $\implies$ No common base class
  \item Only the call signatures (parameter and return types) are the
    same
  \end{itemize}

  \textbf{What does this mean?}

  \begin{itemize}
  \item Perfect for \texttt{<algorithm>} which is also designed for
    speed
  \item Have to be careful when code size is important
  \item Client code has to be instantiated with the type
  \item \textbf{Tradeoff}: speed, code size, elegance, design, taste
    ...
  \end{itemize}
    
\end{frame}
